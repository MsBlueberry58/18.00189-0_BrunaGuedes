\begin{document}

\maketitle
\pagebreak

% 1 =============================================
\section*{Questão 1}

São sentenças as frases: \textit{a, b, c, g}

% 2 =============================================
\section*{Questão 2}

\begin{enumerate}[wide, labelindent=0pt, label=\textbf{\alph*)}]
    % a
    \item   Antecedente – Quantidade suficiente de água\\
    \indent Consequente – Crescimento sadio das plantas
    % b
    \item   Antecedente – Crescimento da oferta de computadores\\
    \indent Consequente – Desenvolvimento científico
    % c
    \item   Antecedente – Programa alterado\\
    \indent Consequente – Novos erros
    %d
    \item   Antecedente – Economia de combustível ou todas as janelas\\
    \indent Consequente – Bom isolamento
\end{enumerate}

% 3 =============================================
\section*{Questão 3}

\newcommand\fA{Rosas são vermelhas }
\newcommand\fa{rosas são vermelhas }
\newcommand\fnA{Rosas não são vermelhas }
\newcommand\fna{rosas não são vermelhas }
\newcommand\fB{Violetas são azuis }
\newcommand\fb{violetas são azuis }
\newcommand\fnB{Violetas não são azuis }
\newcommand\fnb{violetas não são azuis }
\newcommand\fC{Açúcar é doce }
\newcommand\fc{açúcar é doce }
\newcommand\fnC{Açúcar não é doce }
\newcommand\fnc{açúcar não é doce }

\begin{enumerate}[wide, labelindent=0pt, label=\textbf{\alph*)}]
    % a
    \item \fB ou \fnc
    % b
    \item \fC e \fna equivalem ao \fb
    % c
    \item $\neg (B \wedge \neg C) \leftrightarrow
    \neg B \vee C$ (de Morgan)\\
         \indent \fnB ou \fc implicam que \fa
    % d      
    \item \fA ou \fb, e \fnc
    % e
    \item \fnB, ou \fa implica que \fc
    % f
    \item \fC, e \fna é equivalente a \fb
    % g
    \item \fA, ou \fb e \fnc
\end{enumerate}

% 4 =============================================
\section*{Questão 4}

\newcommand\vv{\textbf{verdadeiro}}
\newcommand\ff{\textbf{falso}}

% a ---------------------------------------------
\noindent
\textbf{a)} $[\neg B \wedge (A \rightarrow B)] \rightarrow \neg A$

\medskip
\noindent
$\neg B \wedge (A \rightarrow B)$ é \vv

\noindent
$\neg A$ é \ff

\medskip
\noindent
\dotfill
\medskip

\noindent
$\neg B$ é \vv, logo B é \ff

\medskip
\begin{tabular}{|r l|}
\hline
    \textbf{A} & \vv \\
    \textbf{B} & \ff \\
\hline
\end{tabular}

\medskip
\noindent
\dotfill
\medskip

\noindent
$A \rightarrow B$ é \vv\ e B é \ff, logo A é \ff

\medskip
\begin{tabular}{|r l|}
\hline
    \textbf{A} & \ff \\
    \textbf{B} & \ff \\
\hline
\end{tabular}

\medskip
\noindent
\dotfill
\medskip

\textbf{$\therefore$ é uma tautologia.}

% b ---------------------------------------------
\bigskip
\noindent
\textbf{b)} $[(A \rightarrow B) \wedge A] \rightarrow B$

\medskip
\noindent
$(A \rightarrow B) \wedge A$ é \vv

\noindent
B é \ff

\medskip
\noindent
\dotfill
\medskip

\noindent
Se $(A \rightarrow B) \wedge A$ é \vv, então A é \vv

\medskip
\begin{tabular}{|r l|}
\hline
    \textbf{A} & \vv \\
    \textbf{B} & \ff \\
\hline
\end{tabular}

\medskip
\noindent
\dotfill
\medskip

\noindent
Se $A \rightarrow B$ deve ser \vv\ e B é \ff, A é \ff

\medskip
\begin{tabular}{|r l|}
\hline
    \textbf{A} & \ff \\
    \textbf{B} & \ff \\
\hline
\end{tabular}

\medskip
\noindent
\dotfill
\medskip

\noindent
\textbf{$\therefore$ é uma tautologia.}

% c ---------------------------------------------
\bigskip
\noindent
\textbf{c)} $(A \vee B) \wedge \neg A \rightarrow B$

\medskip
\noindent
$(A \vee B) \wedge \neg A$ é \vv

\noindent
B é \ff

\medskip
\noindent
\dotfill
\medskip

\noindent
Se $(A \vee B) \wedge \neg A$ é \vv\ e B é \ff, A é \vv

\medskip
\begin{tabular}{|r l|}
\hline
    \textbf{A} & \vv \\
    \textbf{B} & \ff \\
\hline
\end{tabular}

\medskip
\noindent
\dotfill
\medskip

\noindent
Se $(A \vee B) \wedge \neg A$ é \vv\, $\neg A$ é \vv, logo A é \ff

\medskip
\begin{tabular}{|r l|}
\hline
    \textbf{A} & \ff \\
    \textbf{B} & \ff \\
\hline
\end{tabular}

\medskip
\noindent
\dotfill
\medskip

\noindent
\textbf{$\therefore$ é uma tautologia.}

% d ---------------------------------------------
\bigskip
\noindent
\textbf{d)} $(A \wedge B) \wedge \neg B \rightarrow A$

\medskip
\noindent
$(A \wedge B) \wedge \neg B$ é \vv

\noindent
A é \ff

\medskip
\noindent
\dotfill
\medskip

\noindent
Se $(A \wedge B) \wedge \neg B$ é \vv, B é \vv

\noindent
Porém, $\neg B$ também deve ser \vv, logo B é \ff

\noindent
\textbf{$\therefore$ é uma tautologia}

% 5 =============================================
\section*{Questão 5}
% a ---------------------------------------------
\bigskip
\noindent
\textbf{a)}
C: Colheita é boa

A: Água suficiente

H: Bastante chuva

S: Bastante Sol

\medskip
$(C \wedge \neg A) \wedge [(H \vee \neg S) \rightarrow A] \rightarrow (C \wedge S)$

% b ---------------------------------------------
\bigskip
\noindent
\textbf{b)}
R: Rússia tinha um poder superior

F: França seria forte

N: Napoleão cometeu um erro

E: Exército falhou

\medskip
$(R \vee \neg F \vee N) \vee (\neg N \wedge (\neg E \rightarrow F)) \rightarrow (E \wedge R)$

% c ---------------------------------------------
\bigskip
\noindent
\textbf{c)}
T: Taxas de eletricidade subiram

C: Consumo diminuirá

U: Novas usinas serão construídas

Co: Contas serão atrasadas

\medskip
$\neg (T \rightarrow C) \wedge \neg (U \vee \neg Co) \rightarrow (\neg C \wedge Co)$

% d ---------------------------------------------
\bigskip
\noindent
\textbf{d)}
J: José pegou as joias

M: Krasov mentiu

C: Ocorreu um crime

E: Krasov estava na cidade

\medskip
$((J\vee M) \rightarrow C) \wedge (C \rightarrow E) \wedge \neg E$

% 6 =============================================
\section*{Questão 6}

\begingroup
\renewcommand{\arraystretch}{1.75}
\begin{table}[h!]
    \centering
    \begin{tabular}{||c|c|c|c|c|c||}
        \hline
        \textbf{p} & \textbf{q} & $\mathbf{p \rightarrow q}$ & $\mathbf{\neg p}$ & $\mathbf{(p \rightarrow q) \wedge (\neg p)}$ & $\mathbf{(p \rightarrow q) \wedge (\neg p) \rightarrow \neg q}$\\
        \hline
        \hline
        V & V & V & F & F & V \\ \hline
        V & F & F & F & F & V \\ \hline
        \textbf{F} & \textbf{V} & \textbf{V} & \textbf{V} & \textbf{V} & \textbf{F} \\ \hline
        F & F & V & V & V & V \\ \hline
    \end{tabular}
\end{table}
\endgroup

% 7 =============================================
\section*{Questão 7}

\begin{align*}
    1. \hspace{10pt} & p \rightarrow q & (hip.) \\
    2. \hspace{10pt} & \neg r \vee (\neg t \vee u) & (hip.) \\
    3. \hspace{10pt} & p \wedge t & (hip.) \\
    4. \hspace{10pt} & q \rightarrow (r \wedge s) & (hip.) \\
    5. \hspace{10pt} & q & (1, hip) \\
    6. \hspace{10pt} & p, t & (3, sim) \\
    7. \hspace{10pt} & r \wedge s & (4, mp) \\
    8. \hspace{10pt} & r, s & (7, sim) \\
    9. \hspace{10pt} & u \qed & (6, 8, add)
\end{align*}














\end{document}
